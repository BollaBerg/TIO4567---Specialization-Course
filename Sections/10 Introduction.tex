\section{Introduction}
In the world of academia, researchers face the issue of "publish or perish", a requirement that they must continuously produce articles, books and other output in order to be able to advance their careers \parencite{Grimes_2018}. This has, in turn, led to an explosion of published papers, in turn making both researchers and the informed public have to wade through mountains of possibly relevant papers in order to get new insights \parencite{Bornmann_2015}. Another result of this overwhelming amount of research is a strict prioritization from publishers, with high ranking journals opting towards publishing novel results rather than replicating studies \parencite{Grimes_2018}.

Two issues arise from this paper explosion. First, the focus on novel results, rather than replicating studies, have led to a "reproducibility crisis" \parencite{Begley_2015}, where published studies are hard to replicate. An example of this can be seen in preclinical research -- a field where one would assume accurate, reproducible results would be of outmost importance -- where \textcite[p.1]{Begley_2015} report that an estimated 75\%-90\% of findings presented in high ranking journals are not reproducible. Such findings are seen across several fields, including psychology, chemistry, biology, physics, earth sciences \parencite{Baker_2016}, and cancer research \parencite{fMRI_no_author}. % As replicating research is the only way to prove it true, the reproducibility crisis could prove fatal to the general trustworthiness of academia.

The second issue arising from the increasingly increasing number of papers is the general speed of research. As the number of total publications has increased exponentially since the 1600s \parencite{Bornmann_2015}, researchers entering a new field are faced with a massive wall of possibly relevant research to review, before they can contribute their own. Although several guides for undertaking this exists (i.e., \cite{Okoli_2015,Popenoe_2021,Nightingale_2009}), they provide aid for the symptoms of the underlying stack of articles, rather than attack the actual problem at hand. Additionally, an ever-expanding volume of papers within a field makes it hard for established researchers to keep up, potentially limiting the collective knowledge they can provide \parencite{Davies_2017}.

One suggestion to combat the issue of reproducibility, mentioned by \textcite{Baker_2016}, is pre-registration, where hypotheses and plans for data evaluation are submitted to a third party before the actual experiments take place. To help with the issue of the overwhelming number of papers, there should be a way for researchers to easily get an overview of a given paper, in order to quickly be able to assess whether it will be relevant for their own research. These solutions can be combined in the way of research questions. When included in a clear way, research questions represent the goals of the presented research, succinctly summarizing the research being done. At the same time, research questions can be used as an easy-to-implement source for pre-registration, giving the same benefits as presented by \textcite{Baker_2016}. This shows how the use of research questions has potential for mitigating the problem of "publish or perish".

% One way of getting an overview of a paper is by looking at the goals of the research being done. Several guides for academic writing (i.e., \cite{Rosenfeldt_2000,Jha_2014,Lin_2012,Davidson_2012,Cuschieri_2019}) recommends including such goals in the form of research questions -- short, direct questions representing what the researchers are trying to learn through their research. As research questions succinctly summarize the research being done, they present a potential for mitigating the problem of "publish or perish".
% One suggestion to combat the issue of reproducibility, mentioned by \textcite{Baker_2016}, is pre-registration, where hypotheses and plans for data evaluation are submitted to a third party before the actual experiments take place. As this requires a significant change in the overall system built around existing researchers, this paper instead proposes a softer version of the solution -- submitting, either to third parties or internally, clearly formulated research questions ahead of the actual research.
%  This goes towards helping two issues. First, providing research questions ahead of work can be considered a form of pre-registration, and as such provides similar benefits to the established method -- \mbox{researchers}, having already submitted the research questions they are looking at, are prevented from cherry-picking attractive results, instead being guided to work towards their previously defined goals. Secondly, clearly establishing research questions early in an article, preferably in the abstract or introduction, simplifies literature reviews, as reviewers can quickly see whether the article will be relevant or not for their own research. Including research questions in the introduction or abstract is recommended in several existing guides for academic writing (i.e., \cite{Rosenfeldt_2000,Jha_2014,Lin_2012,Davidson_2012,Cuschieri_2019}).

With this background, this paper will look at the following two research questions:
\begin{enumerate}
    \item How are research questions currently used, and to what degree are they clearly presented, in existing papers published in the field of responsible artificial intelligence?
    % \item How should research questions be presented, in order to ensure maximum readability and clarity?
    \item How can active use of research questions mitigate the problem of "publish or perish"?
\end{enumerate}

The problem of "publish or perish" has been widely discussed, with several attempts at finding solutions (i.e., \cite{Grimes_2018,Davies_2017,Bornmann_2015}). This paper aims to contribute to mediating the problem, by providing ways to ensure future papers are as easy-to-understand and well structured as possible. To achieve this goal, three peer-reviewed and published papers are analyzed. The papers are carefully selected to be representative of the field of responsible artificial intelligence, and are evaluated against a pre-defined set of criteria, focusing on article structure and -clarity. This evaluation is then used as background for a discussion of research question use, and how good use of research questions can help mitigate the problems arising from an ever-expanding volume of papers. As "publish or perish" is unlikely to go away in the near future, this discussion is highly relevant, and the solution presented in this paper may be used as one of many tools to mediate the problem.

The paper will start by discussing the methods used for selecting the papers and the evaluation criteria. Then, the criteria used for evaluating the papers are presented, before each paper goes through the same process -- a short presentation, followed by an in-depth analysis of the paper. Any findings are then used as foundation for a discussion, where research question use is discussed in-depth, including advice for how to ensure maximum clarity in future articles.

This paper contributes to mitigating the problem of "publish or perish" by creating a short, easy-to-implement guide for using research questions in academic papers.