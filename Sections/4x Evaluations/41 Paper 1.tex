% Paper 1
\subsection[Presentation of Barredo Arrieta et al. (2020)]{Presentation of \textcite{Barredo_2020}}
The first paper, \citetitle{Barredo_2020} by \textcite{Barredo_2020}, is a conceptual paper explaining the field of explainable artificial intelligence (XAI). The paper starts out by clarifying concepts and definitions for XAI. It introduces the concept of \textit{audience} to XAI, changing the definition of explainable AI to "Given an audience, an explainable Artificial Intelligence is one that produces details or reasons to make its functioning clear or easy to understand" \parencite[p.85]{Barredo_2020}.

The authors goes on to discuss current methods for making artificial intelligence systems more explainable, before it performs a thorough literature review in the field of XAI, splitting the current research in two parts -- one focused on general explanation methods, the other focused solely on methods aimed at explaining deep learning systems. With the review fresh in memory, the paper discusses challenges and future research opportunities within the field of XAI, mentioning cases such as interpretability possibly lowering performance, the lack of consensus regarding the definition of explainability, and the current lack of methods for explaining deep learning models.

The paper goes on to introduce the concept of responsible artificial intelligence, by building upon the research done on XAI, adding fairness, accountability and privacy to the previously used definition. The paper rounds out by welcoming reflections on XAI in sensitive scenarios, finishing by discussing how XAI methods can work in privacy- and security-focused systems.



\subsection[Evaluation of Barredo Arrieta et al. (2020)]{Evaluation of \textcite{Barredo_2020}}
While it contains novel definitions, exhaustive literature reviews and important insights within the field of responsible artificial intelligence, \textcite{Barredo_2020} does not contain any concrete research questions. Instead, the aim of the research is covered in a single sentence, where it is stated that "[...] this overview aims to cover the creation of a complete unified framework of categories and concepts that allow for scrutiny and understanding of the field of XAI methods" \parencite[p.83]{Barredo_2020}. Including the aims of the research as part of a body of text, rather than explicitly listed, aligns with recommendations from \textcite[p.359]{Davidson_2012}, although with a requirement that "At the end of the Introduction, the reader should have a clear idea of what the research question is, why it is important and what the investigators aimed to do in the study" \parencite[p.360]{Davidson_2012}.

In addition to the single-sentence summary of the aims of the article, \textcite{Barredo_2020} rounds out the introductory chapter with a comprehensive list of all contributions of the article, written as a numbered list. This way of summarizing the contributions of an article, rather than the goals, makes it represent the actual findings of the research. As such, it more closely aligns with the conclusion, rather than the introduction. This difference in alignment is explicitly mentioned by both \textcite[p.2]{Jha_2014} and \textcite[p.83]{Lin_2012}, who recommends to "[g]ive in the introduction only the strictly pertinent references and do not include the data or conclusions from the work being reported"  and to "[...] not include any of the data or conclusions from your study in the Introduction", respectively.

As mentioned in \autoref{sec:introduction}, the problem of "publish or perish" is twofold, split between reproducibility and an increasing number of papers. Including a summary of contributions in the introduction, as done by \textcite{Barredo_2020}, does not contribute to solving or mitigating the first part of the problem, as it does not aid reproducibility. It can, however, mitigate the latter part of the problem, as including the contributions in the introduction allows a reader to quickly assess whether the article is relevant for their own research. In fact, including a summary of the results and conclusions in the abstract is recommended by both \textcites[p.2]{Jha_2014}[p.82]{Lin_2012}[p.114]{Cuschieri_2019}[p.359]{Davidson_2012}[p.115]{Katz_2006}[p.85]{Rosenfeldt_2000}. Although a short summary is already part of the abstract of \textcite[p.82]{Barredo_2020}, the complete summary of contributions is longer, and goes in greater detail on what the article brings to the field of responsible artificial intelligence. As most abstracts are given strict word limitations \parencite[p.85]{Rosenfeldt_2000}, the authors may have opted to include the summary in the introduction to be able to fully point out their contributions.

% (How does the results -- or rather, the discussion points, as there are no result chapter -- connect with contributions?)
The article largely sticks close to its original aim, mentioned above. Section 2 of \textcite{Barredo_2020} explains categories related to explainable AI and connects definitions found in other papers to their own definition of XAI. Section 3  and 4 connects existing concepts and methods within the field of XAI to the work done in section 2, while section 5 is focused on the status of the research being done in the field of XAI, fulfilling the "allow for scrutiny"-part of the aim. The only parts of the paper that does not directly connect to the stated aim of the article is section 6, which instead focuses on principles for responsible artificial intelligence. It can be argued that this is relevant to the aim of "understanding of the field of XAI methods", something the authors do in their introduction of the section, stating that "[section 6] presents some of the most important and widely recognized principles [of responsible artificial intelligence] in order to link XAI –- which normally appears inside its own principle –- to all of them" \parencite[p.103]{Barredo_2020}. However, as the primary focus of the section shifts from XAI to responsible artificial intelligence, the final section of the paper moves outside of the stated research goal.
% If needed for the aim:  \parencite[p.83]{Barredo_2020}

% (How does the conclusion connect with goal/contribution?)
The conclusion sticks relatively closely to the research goal, as it follows the same topics as discussed in the rest of the article. Most of it summarizes the work on AI, discussing the work done and key contributions within each section. While one paragraph touches on responsible artificial intelligence, thus moving slightly outside of the stated research goal, the main focus is on how XAI is a core component of any responsible artificial intelligence system, thus connecting it back with the stated aim for the paper.

% (Does the article point towards the goal/contributions at any point?)
While the content of the paper aligns with the stated goal, \textcite{Barredo_2020} contains no explicit references to it, instead connecting their work to their goal through the overarching topic of XAI. An example of this can be seen in the introduction of section 5, where the authors state that the section will be used to "[...] revisit [the challenges] and explore new research opportunities for XAI, identifying possible research paths that can be followed" \parencite[p.99]{Barredo_2020}. While the quote makes no explicit references to the research goal, it is connected through the topic of XAI. This same method of connecting through the topic can also be seen in the conclusion, where several references are made to the topic -- such as "Implications of XAI in fairness have also been discussed in detail" and "[Our reflections] agree on the compelling need for a proper understanding of the potentiality and caveats opened up by XAI techniques" (both from \cite[p.108]{Barredo_2020}). Neither of these quotes mention the aim of the paper, but it is still clearly connected through the topic.

This way of connecting the research to the research goal goes against several of the selected writing guides. \textcite[p.115]{Cuschieri_2019} states that "The factual answer to the research question that was described in the ‘Introduction’ section needs to be clearly illustrated" and "The discussion must aim to answer the research questions that were posed in the ‘Introduction’ section". \textcite[p.360]{Davidson_2012} recommends to "Start the Discussion with the answer to the research question", and \textcite[p.2]{Jha_2014} argues to "Begin the discussion with brief recapitulation of the main findings (the answer to the research question)". Finally, \textcite[p.84]{Lin_2012} argues that "the first paragraph [of the discussion] should discuss the major findings and state whether the hypotheses were supported or rejected." Although the aim of \textcite{Barredo_2020} is not formulated as something that can be supported or rejected, the point still stands that the discussion should explicitly discuss the research goal. To summarize, most current guides for academic writing suggests to explicitly connect the results and discussion to the given research goal, something \textcite{Barredo_2020} fails to do.

% Conclusion
In conclusion, \textcite{Barredo_2020} does not explicitly state their research questions, instead including the aim of their research as part of a paragraph in their introduction. While the aim could be expressed more explicitly or highlighted in their introduction, the authors stick to the original goal of their paper, largely aligning their work with the stated aim. The goal is used throughout most of the paper, with four of five sections, as well as the full concluding chapter, sticking close to it. While they do not explicitly connect their research to the stated goal, this connection is made through the overarching topic of XAI, ensuring an alignment between the research and the research goal.