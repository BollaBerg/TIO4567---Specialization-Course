% Paper 2
\subsection[Presentation of Vakkuri et al. (2020)]{Presentation of \textcite{Vakkuri_2020}}
The second paper, \citetitle{Vakkuri_2020} by \textcite{Vakkuri_2020}, is a quantitative paper looking at how and to what extend AI ethics are applied within the AI industry. The paper starts by describing the attractiveness of artificial intelligence, and how this has contributed, and is likely to continue contributing, to an increasing number of ethical incidents involving the technology, before a brief discussion on how their survey -- using respondent data to analyze the practice of AI ethics -- is novel and has never before been done.

The paper contains data gathered from 249 employees from 211 companies, primarily located in the US or Finland \parencite[p.52]{Vakkuri_2020}. Over 65 per cent of the respondents answered that they were able to influence the functionality of their systems, and of the responding companies, over half were actively developing artificial intelligence systems. The authors note that the answers were similar whether the employing company were developing artificial intelligence or not, and independent of the geographic location of the company. All responses are therefore included in the paper.

\textcite{Vakkuri_2020} rounds out the paper by discussing what the presented results means for other software development companies. They connect AI ethics to other ethical trends, such as data privacy and ecological issues, and briefly discuss how AI ethics can create business advantages. The authors go on to mention several tools for implementing AI ethics in a company, such as existing guidelines, but point out that there is limited access to general use, off-the-shelf solutions for AI ethics. Finally, the paper goes on to mention several anti-patterns companies looking to implement AI ethics should avoid -- such as outsourcing ethics or delegating ethics work to a single individual -- before rounding out by a discussion of how the evolution of AI ethics is likely to impact the software development industry.

\subsection[Evaluation of Vakkuri et al. (2020)]{Evaluation of \textcite{Vakkuri_2020}}
% Does the paper contains explicit research questions? What are they? If not, what is the goal of the research?


% How can this paper's use of RQs contribute to mitigating "publish or perish"?

% (How does the results connect with RQs?)

% (How does the conclusion connect with RQs?)

% (Does the article explicitly point reference the RQs at any point?)

% Conclusion
