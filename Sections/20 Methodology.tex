\section{Methodology}

\subsection{Method for selection of papers}
\label{sec:paper-selection}
As the scope of this evaluation is limited to three papers, great care was taken when making the selection, to ensure as generalizable results as possible. To ensure optimal breadth in the selection, papers were chosen to have different levels of recognition, from different authors in different years, yet in the same field of science -- responsible artificial intelligence. Additionally, papers were chosen to ensure both conceptual, qualitative and quantitative papers were part of the evaluation.

For the first paper, the goal was to have a relatively recent paper, with a good standing in the field of responsible artificial intelligence. A search was conducted on the Scopus digital database of papers using the terms "Responsible AI" and "ethical AI". 404 papers were retrieved, and to ensure selected papers are relevant in the field, all results were sorted from most to least cited. The paper \citetitle{Barredo_2020} by \textcite{Barredo_2020}, published in the journal \textit{Information Fusion} was at the top of the list with 1447 citations. Being of a conceptual nature, highly cited and recent, the article made for a good first paper.

For the second paper, the goal was to find a quantitative study done within the field of responsible artificial intelligence. Although AI is in itself a highly quantitative field, finding quantitative papers turned out to be harder than expected, and the 29 most cited papers in the above-mentioned Scopus search were conceptual. The 30th most cited, however, was of a quantitative nature. The selected paper was \citetitle{Vakkuri_2020}, written by \textcite{Vakkuri_2020}. The paper, published in \textit{IEEE Software}, has 16 citations on Scopus. While this is a relatively low number, it is still of some relevancy in a relatively small field as responsible artificial intelligence, and provides a good breadth when analyzed alongside \textcite{Barredo_2020}.

For the final paper, the goal was to find a qualitative study within the field. The third most cited paper in the Scopus search was a literature review, and thus satisfied the requirement for qualitativeness. The third paper, then, was chosen to be \citetitle{Ntoutsi_2020}, written by \textcite{Ntoutsi_2020}. Published in \textit{Wiley Interdisciplinary Reviews: Data Mining and Knowledge Discovery}, the paper has 109 citations on Scopus, placing it nicely between the two first papers.

The complete search history used to find the above-mentioned articles can be seen in \autoref{app:paper-search}. Note that the journey was far from straight-forward, as several fields of science failed to show promising results, or included too many, too old or too narrow-focused articles to have any significant relevance.

The field of responsible artificial intelligence was chosen for two reasons. First, it is a relatively popular field of science, with frequent additions to the state of research. Secondly, it is a field closely aligned with the author's upcoming Master's thesis, aligning it with both personal and academic interests. One issue with this field, however, is the relative freshness of the research field. As most research is done relatively recently, there is a lack of spread of papers over time. This is likely to not significantly impact the results of this analysis.

\subsection{Method for selection of assessment criteria}
\label{sec:criteria-selection}
A literature review was conducted in order to find a set of well-fitting assessment criteria for the topic. The search was done using the Scopus electronic database and Google Scholar, and was filtered on articles that were either open access or available through NTNU. The search terms are listed in \autoref{app:criteria-search}.

A challenge when searching for articles discussing research quality -- and by extension contributing to the selected assessment criteria -- was that most of the results were either aimed at users of the research (i.e., \cite{Mayhew_1993,Domholdt_1985}), reviews done on other, unconnected fields of science containing few generalizable guidelines (i.e., \cite{Wade_2004,Biddle_2011,Polanczyk_2015}), or research done on image quality (i.e., \cite{Wang_2002,Mittal_2012}). To mitigate this, a thorough review of each paper was done, to ensure that all assessment papers contain usable criteria for research quality or guidelines for writing or reviewing academic papers.

The result of the literature review was a selection of seven papers. Six of these \parencite{Rosenfeldt_2000,Jha_2014,Lin_2012,Davidson_2012,Cuschieri_2019,Katz_2006} were guides for academic writing, providing a source of guidelines that the articles could be compared against. The remaining paper \parencite{Martenson_2016} was a guide for how to evaluate academic papers. While primarily targeted towards reviewers, especially peer reviewers, the article still provide ways to evaluate the selected papers, especially because it gives an idea of what peer reviewers are supposed to look for when evaluating papers.

The selected topic for this paper revolves around the actual structure of the selected papers, rather than the methodology used in the papers. As such, the same assessment criteria are used for papers using both quantitative and qualitative methods.

\subsection{Limitations}
There are several limitations with these methods for selecting papers. First of all, the small scope of the course and this paper provides a limit for how in-depth the literature searches can go. As an exhaustive literature review could take weeks, if not months, the depth of this review ended up being much shallower, with only a selected number of papers being reviewed. While these were selected as the most promising articles based on the available metadata, there is a significant risk that useful or interesting articles may have been skipped due to this.

The same limitation also applies to the search for criteria-relevant papers. While several guides for academic writing were found, only one evaluation guide, and no other relevant sources for criteria, was found. While it would take more time than is assigned for this course, a deeper literature search would likely provide more relevant papers, thus leading to more relevant assessment criteria. Missing these may potentially lead to some relevant assessment criteria being missed, weakening the criteria presented in \autoref{sec:criteria}.

One limitation arises from the chosen field of science. While artificial intelligence itself is a relatively established field, dating back to at least the 1950s \parencite{Moor_2006}, responsible artificial intelligence is relatively new. This leads to both a narrow spread of publication dates, as all selected papers are from 2020, as well as fewer non-conceptual papers. As most articles within the field are published relatively recently, the field has barely begun to move past the conceptualizing stage, which greatly limits the amount of qualitative and non-conceptual quantitative work that can be done.

The age of the field may also lead to less generalizable results. As a relatively new field of science, it is natural to assume that it will attract younger researchers looking to make significant contributions in the field's youth. This may lead to a lower quality of papers, and different issues, compared to more established fields of science, where researchers are more likely to have more experience with writing papers.