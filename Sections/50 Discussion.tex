\section{Discussion}
The recommendations from the selected writing- and reviewing guides are unified -- "Research emanates from at least one [research question]" \parencite[p.596]{Martenson_2016}, "The ‘objective’ section needs to contain a clear statement of the research question/s" \parencite[p.115]{Cuschieri_2019} and "Before writing the paper, think carefully about what the research question is, and when writing keep the paper focused on the question." \parencite[p.357]{Davidson_2012}, to quote a few -- in that research questions are essential for writing academic papers. Yet, interestingly, none of the evaluated papers included their research questions, despite being approved by established journals. This section will discuss potential reasons for this, as well as ways to ensure future papers include their research questions, as doing so will not only ensure compliance with recommendations for academic writing, but also, more importantly, can be used as a way to mitigate the problem of "publish or perish", as discussed in \autoref{sec:introduction}.


\subsection{Reasons for failure to include research questions}
\label{sec:discussion-reason}
There may be several reasons why all three papers failed to include their research questions. In the author guidelines of the three journals the evaluated papers were published in, neither \textcite{Elsevier_guide}, where \textcite{Barredo_2020} was published, \textcite{IEEE_guide}, where \textcite{Vakkuri_2020} was published, nor \textcite{WIREs_guide}, where \textcite{Ntoutsi_2020} was published, requires authors to include their research questions when submitting papers. This leaves us with a simple explanation for their exclusion -- papers have a length limit \parencite{IEEE_guide}, and including content that is lengthy but not required limits the amount of novel and interesting content that can be included. With this explanation, the responsibility for authors not including their research questions falls on the journals, as they create the requirements for the submitted papers. Future research should look at compare author guidelines in different journals, and check to what degree their differences affect the accepted papers. If the explanation is found to be right, journal editors should carefully consider whether they should make adjustments to their guidelines, such as by requiring research questions as part of the introduction, or if the current guidelines produce the best results possible.

The missing compliance between the recommendations and the submitted papers may also be due to academic writing guides being outdated. All papers included the overarching goal of their research, with \textcite{Barredo_2020} additionally including a summarizing list of their contributions. The authors may have felt like this was enough to "describe the aim of the study" \parencite[p.359]{Davidson_2012}, explain "the purpose for conducting the research" \parencite[p.115]{Cuschieri_2019} and show they have "[identified] and have the idea that [they] wants to communicate through the paper" \parencite[p.81]{Lin_2012}, as the various guides recommend. To assess whether this holds true, future research can employ quantitative methods, to evaluate larger volumes of papers than these three. If this is the case, and modern researchers prefer other ways of representing their research goals, new guides for academic writing should be constructed. This process should ideally be conducted by experts on academic writing, in collaboration with editors of major journals, to create accurate, relevant and generalizable guides for future papers.


\subsection{Research questions and quality}
Although several models for assessing the quality of an academic paper has been created (i.e, \cite{rubin_2011,Robey_1998,Klein_1999}), there is still a "lack of widely acknowledged quality standards," and "a need for determining a universal concept model for the quality of research practice" \parencite[p.595]{Martenson_2016}. As such, drawing exact connections between the inclusion of research questions and quality is challenging.

Still, the included guides for academic writing are clear that research question are essential for writing good papers. \textcite[p.114]{Cuschieri_2019} states that an author "needs to have a clear vision of the aim and scopes of [their] paper" in order to "ensure the execution of a high quality paper that is likely to be accepted for publication." Likewise, \textcite[p.9]{Barroga_2022} states that "it is crucial to have excellent research questions to generate superior hypotheses," which will "determine the research objectives and the design of the study, and ultimately, the outcome of the research." Including explicit research questions can be seen as a formalized method for stating the aim of a paper, thus contributing to the quality of the result. Likewise, including research questions in a paper contribute to increase the readability, and increases the paper's alignment with current guides for academic writing.


\subsection{Methods to ensure maximum quality of papers}
Depending on which of the theories presented in \autoref{sec:discussion-reason} are found to be correct, there are several methods that can be employed to ensure the quality of academic papers is as high as possible. First, journals can change their author guidelines, to recommend -- or possibly force -- authors to include research questions in their submitted papers. As this paper shows a connection between research question use and readability, and thus paper quality, doing so would ensure the quality stays as high as possible.

Journals and academic institutions have the option of going one step further than this, requiring authors to register their research questions ahead of starting their research. This aligns with the suggestions put forth by \textcite{Baker_2016}, and thereby contributes twofold. First, pre-registering research questions could ensure researchers avoid nit-picking results, and thus increase reproducibility and lead to accurate and honest results. Secondly, such a pre-registration would ensure research questions are available when submitting papers, thus aiding the readability and quality of the submitted paper.

Finally, researchers, journal editors and experts on academic writing could consider alternative structures for academic papers. Instead of listing their research questions, \textcite{Barredo_2020} finishes their introduction with a list of their contributions. While these contributions can only be seen after the research has finished, and as such are not available for pre-registering, including contributions in a structure way leads to the same readability benefits as mentioned above, and makes it easy for readers to quickly assess whether the paper is relevant for their own research. In fact, \textcite{Chen_2022} argues that contributions are the most important part of any research. To further explore this method, future research should try to find alternative paper structures than what is currently employed within academia, and evaluate whether such alternative structures can increase the readability and quality of academic papers.


Although the use of research questions was similar between the quantitative and qualitative papers evaluated in this paper, there could be differences between the methodologies when it comes to the optimal use of such questions. \textcite[p.3-4]{Barroga_2022} states that quantitative research questions are "precise" and "usually framed at the start of the study," while qualitative research questions "broadly explore a complex set of factors" and "are usually continuously reviewed and reformulated." Based on this, it is natural to assume that clear research questions possibly are more important in quantitative works -- where they can ensure that the researchers do not adjust their goals after seeing the data, thus cherry-picking results -- compared to qualitative works, where researchers are typically looking for connections and patterns and the research questions are more related to the topic of research, rather than concrete connections. Future research should compare differences in the use of research questions, and whether such use impacts the quality of quantitative and qualitative papers differently. The results from such findings may be used to develop separate guides for the different methodologies, to ensure maximum quality for all papers, no matter the method used.